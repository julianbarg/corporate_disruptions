%\section{Introduction}
Corporate approaches to environmental management often follow the ISO 14001 standard, or similar models that adheres to the \textit{Plan-Do-Check-Act} (PDCA) approach \citep{ISO2015}. When following this model, managers try to discover the risks they face in the future, asses them, and manage them accordingly, repeated ad infinitum (or ad nauseam). The reiteration accounts for the fact that the managers obtain information about the new status of the world as history moves forward. Still this approach carries in it the deceiving perception that we only need to manage risks we know. Ulrich Beck instead suggests that "in the [modern] \textit{risk society}[emphasis by this author], unknown and unintended consequences come to be the a dominant force in history and society"\citep[p. 22]{Beck1992}.

These unknown and unintended consequences take different forms. We know emitting greenhouse gases carries risk, but we do not know the specific consequences that will come into reality (i.e., the specific impact on the climate in a specific location at a specific point in time). Beck introduces the risks which the risk society faces as much more \textit{wicked}: they include events such as atomic accidents, large-scale industrial disasters, or global environmental pollution from presumably safe chemicals. Management literature has so far captured this debate in three discourses. There is a discourse on \textit{grand challenges} which tries to open our eyes to some of the more difficult to cognize future challenges \citep{George2016, Howard-Grenville2014}. Robust action is closely related to grand challenges and describes actors' decision making in the face of uncertainty \citep{Ferraro2015}. Finally, resilience can help organizations to resist potentially devastating changes \citep{VanderVegt2015, Flammer2017}.

Temporality on the firm level has been suggested to be a potential key element to all three approaches \citep{Bansal2014, Bansal2019, Kunisch2017}. Temporality can take many different forms, with the most commonly researched in management literature being long-term vs short-term strategy \citep{Flammer2017, Slawinski2015, Wang2012}, the strategic orientation of agents \citep{Souder2012, Wowak2015, Souder2010, Chen2017}, and the speed of decision making or execution \citep{Baum2003, Kownatzki2013, Dykes2019}. 

In the context of sustainability, temporality usually alludes to a \textit{long-term} outlook on the impacts of business decisions, following the maxim that "business operations should not compromise the welfare of future generations"\citep[p. 531]{Slawinski2015}. \textit{Short-termism} on the other hand includes making trade-offs that benefit the company at the present time or near future at the expense of future well-being (of society, environment, or business), out of necessity or out of opportunism. For instance, a company may choose to pollute local water streams (legally or illegally) to cut cost, endangering the long-term health of both local residents and the environment, while also potentially damaging business relationships\citep{Slawinski2015}. Another relationship between temporality and sustainability is explored in \citet{Kim2019}: the managers in the study experience the present as a "long present". Even when under resource constraints, they would organize their tea plantation to actualize reliable, continuous resource flows, both in the present and in the future, ad perpetuum.

We expand the literatures on temporality and sustainability by





