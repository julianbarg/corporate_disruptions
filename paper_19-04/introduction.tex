%\section{Introduction}
Corporate approaches to environmental management often follow the ISO 14001 standard, or similar models that adheres to the \textit{Plan-Do-Check-Act} (PDCA) approach \citep{ISO2015}. When following this model, managers try to discover the risks they face in the future, asses them, and manage them accordingly, repeated ad infinitum (or ad nauseam). The reiteration accounts for the fact that the managers obtain information about the new status of the world as history moves forward. Still this approach carries in it to some extent the deceiving perception that we only need to manage risks we know. Ulrich Beck instead suggests that "in the [modern] \textit{risk society}[emphasis by this author], unknown and unintended consequences come to be the a dominant force in history and society"\citep[p. 22]{Beck1992}.

These unknown and unintended consequences take different forms. We know emitting greenhouse gases carries risk, but we do not know the specific consequences that will come into reality (i.e., the specific impact on the climate in a specific location at a specific point in time). Beck introduces the risks which the risk society faces as much more \textit{wicked}: they include events such as atomic accidents, large-scale industrial disasters, or global environmental pollution from presumably safe chemicals. Management literature has so far captured this debate in three discourses. There is a discourse on \textit{grand challenges} which tries to open our eyes to some of the more difficult to cognize future challenges \citep{George2016, Howard-Grenville2014}. Robust action is closely related to grand challenges and describes actors' decision making in the face of uncertainty \citep{Ferraro2015}. Finally, resilience can help organizations to resist potentially devastating changes \citep{VanderVegt2015, Flammer2017}.

Temporality on the firm level has been suggested to be a potential key element to all three approaches \citep{Bansal2014, Bansal2019, Kunisch2017}. Temporality can take many different forms, with the most commonly researched in management literature being long-term vs short-term strategy \citep{Flammer2017, Slawinski2015, Wang2012}, the strategic orientation of agents \citep{Souder2012, Wowak2015, Souder2010, Chen2017}, and the speed of decision making or execution \citep{Baum2003, Kownatzki2013, Dykes2019}. 

In the context of sustainability, temporality usually alludes to a \textit{long-term} outlook on the impacts of business decisions, following the maxim that "business operations should not compromise the welfare of future generations"\citep[p. 531]{Slawinski2015}. \textit{Short-termism} on the other hand includes making trade-offs that benefit the company at the present time or near future at the expense of future well-being (of society, environment, or business), out of necessity or out of opportunism. For instance, a company may choose to pollute local water streams (legally or illegally) to cut cost, endangering the long-term health of both local residents and the environment, while also potentially damaging business relationships\citep{Slawinski2015}. Another relationship between temporality and sustainability is explored in \citet{Kim2019}: the managers in the study experience the present as a "long present". Even when under resource constraints, they would organize their tea plantation to actualize reliable, continuous resource flows, both in the present and in the future, ad perpetuum.

We expand the literatures on temporality and sustainability by elaborating on the theme of pace. In strategic change literature, pace describes the speed at which strategic change is initiated or implemented \citep[p. 1026]{Kunisch2017}. Organizations may move at a slow pace, meaning that changes are implemented slowly but thoroughly \citep{Amis2004}. At the same time, the environment may require of an organization to move quickly, e.g., in order to adjust to external shocks that occur in rapid succession. Thus, pace can refer both to the speed of an individual change episode, or to the succession of different episodes. 

In a modern bureaucracy, where decision makers are removed by several, formal levels from activities "on the ground" (e.g., on the shop floor, or throughout the internal supply chain) \citep[p. 147]{DiMaggio1983}, the principal-agent problem makes it difficult for the decision makers to assess the implementation of the individual initiative on the ground. When in this environment major policy changes are initialized at a fast pace, it is possible that old routines on the ground are, intentionally or unintentionally, broken down without new ones being embedded to a satisfactory degree. For instance, personnel might be laid off, or shuffled away from equipment that they are familiar with. Or time slots that are necessary for maintenance of equipment may be unintentionally reallocated to other tasks as a side effect of change episodes. If changes take place at a slow enough pace, personnel on the ground might still be able to take the necessary steps to preserve important routines. But a high pace might render them either unable or unwilling (because of the stress and disgruntlement often involved with change) to take the necessary precautions for changes not to lead to severe negative outcomes.

The PDCA approach to environmental management leans toward a view of the (environmental) manager as the master of the shop floor, who is able to pull complex information from a variety of localities and work processes into a document and make rational decisions to optimize environmental impacts. However, environmental management usually takes place in the context of a businesses that make strategic moves to stay up to date on changes in the external environment. Environmental impacts not only occur regularly, as a part of production processes that involve well-controlled and measured emissions, such as carbon emissions from burning coal that we can statistically estimate from material inputs and the equipment involved. In a dynamic corporate environments, environmental impacts stem from those regular processes as well as from unintended consequences of corporate action. Canadian oil companies make efforts to render tar sands operations less polluting, but incur vast environmental impacts from pipeline leaks. A multinational corporation may engage its suppliers on resource efficiency, only for all improvements to be negated by environmental disasters. And chemical companies strive to improve the efficiency of chemical processes to preserve energy and resources, a major environmental impact of the industry today are ground and water pollution around the globe.

The PDCA approach can work well for regulating known environmental emissions, but it cannot be the answer to the massive environmental impacts the world faces year by year, such as the sever loss of wildlife from the Deepwater Horizon oil spill, the water pollution from the 2005 Jilin chemical disaster, or the widespread ground pollution at the US at superfund sites. Whereas we do not believe that the risk of these pollution events can ever be fully eliminated, we do believe that this branch of environmental management does deserve more attention than it currently receives. As a first step would certainly be research on potential causes of unintended environmental impacts. The largest chemical disaster to date, the Bhopal disaster, was caused by the breakdown of important routines\citep{Trotter1989}, which we have above suggested to be a potential outcome of a high corporate pace.





